{\Large Ultraviolet Light}
\begin{itemize}
\item Ultraviolet light is beyond the range of human vision.

\item Physicists have divided ultraviolet light ranges into Vacuum Ultraviolet (VUV), Extreme Ultraviolet (EUV), Far Ultraviolet (FUV), Medium Ultraviolet (MUV), and Near Ultraviolet (NUV).

\item UV-A, UV-B and UV-C were introduced in the 1930's by the Commission Internationale de l'\'{E}clairage (CIE, International Commission on Illumination) for photobiological spectral bands.

\item Ultraviolet ``blacklights`` used in theatres and nightclubs emits mostly invisible UVA1 light in the 350-380nm range, though a smaller amount of visible purple light is also emitted. Phosphors absorb this UV-A energy and re-emit it as visible light. Naturally occurring phosphors in your teeth and nails cause them to glow under blacklights. Bluing agents in some detergents also contain phosphors. % The natural UV-A in sunlight causes the same absorbption and re-emission of visible light, but this affect isn't noticed in the midst of other bright visible light. %%% <---- need to clean that up

% \item The sun produces a wide range of frequencies including all the ultraviolet light, however, UVB is partially filtered by the ozone layer and UVC is totally filtered out by the earth's atmosphere.

% possibly biased source for most of the below info: https://www.skincancer.org/prevention/uva-and-uvb
\item All UV exposure comes with some health risks.

\item UV-A accounts for 95\% of the UV raditation that makes it through our atmosphere from the sun and is the primary cause of skin-tanning.

\item UV-A is subdivided into UVA1 (340-400nm) and UVA2 (320-340nm), with the latter considered more harmful. The very low energy UVA1 produced by indoor blacklights is not believed to be enough to damage DNA.

\item UV-B only makes it to the surface in small amounts but causes more damage to humans. It does not penetrate the deeper layers of the skin but plays a key role in sunburn, skin reddening, and skin cancer. UV-B is usually blocked by glass.

\item UV-C is the most harmful to humans but is almost entirely filtered out by the ozone layer. Artifically generated UV-C is used as a germicide.

% \item The CIE originally divided UVA and UVB at 315nm, later some photo-dermatologists divided it at 320nm.

\item A bumblebee can see light in the UV-A range which helps them identify certain flowers.
\end{itemize}




% From: http://www.ping.at/cie/publ/abst/134-99.html
% Press Release:
% CIE Collection in Photobiology and Photochemistry, 1999
% CIE 134-1999 ISBN 3 900 734 94 1
% This volume contains short Technical Reports prepared by various Technical Committees within CIE Division 6.

% 134/1: TC 6-26 report: Standardization of the Terms UV-A1, UV-A2 and UV-B
% The terms UV-A, UV-B and UV-C were introduced in the 1930's by CIE Committee 41 on Ultraviolet Radiation as a short-hand notation for photobiological spectral bands. It was never intended that the bands were exclusive for different effects. The bands have been in widespread use in different medical fields and scientific research. UV-A and UV-B were divided at 315 nm by the CIE. In recent decades, some photo-dermatologists and others have used different dividing lines such as 320 nm without recognizing the importance of maintaining an international standardized terminology. Because the terminology is used in many fields, this report recommends that the 315 nm division between UV-A and UV-B be maintained. However, recent research has clearly shown a difference in the photobiological interaction of long and short wavelength UV-A radiation with DNA. This led to a further division of UV-A into UV-A1 and UV-A2 with a dividing line at approximately 340 nm. While this division may be of value, the committee does not recommend officially to split UV-A into these two sub-bands at this time. Further research may justify a dividing line different from 340 nm in the future.
