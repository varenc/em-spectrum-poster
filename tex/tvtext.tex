%TV text


%%% Update this to take into account the analog -> digital transition for TV transmissions
% - UHF tv bands only goes up to channel 38 now!
% - add in digital tv bands?

{\Large Television}
{
\definecolor{HumanAudioColor}{rgb}{0.2,0.2,0.6}
\begin{itemize}

\item Terrestrial broadcast TV uses the VHF and UHF ranges (30MHz - 3GHz)  \psframebox[linestyle=none]{\rput(0.05,.05){\psframebox[fillstyle=solid,framearc=0.25,fillcolor=red]{\textcolor{white}{\tiny 05}}}}\hspace{.1in}.

\item Satellite television is transmitted in the C-band (4 - 8 GHz)
\psframebox[linestyle=none]{\rput(0.05,.05){\psframebox[fillstyle=solid,framearc=.25,fillcolor=green]{\textcolor{black}{\tiny 05}}}}\hspace{.1in} and Ku-band (12 - 18 GHz)  \psframebox[linestyle=none]{\rput(0.04,0.02){\psdots[framesep=1pt,fillcolor=BrightGreen,fillstyle=solid,dotstyle=triangle,linecolor=BrightGreen](0,0)}}\hspace{.08in} where one of many satellites is shown. To eliminate interference, the stations are broadcast in alternating polarities, for example, Ch 1 is vertical and Ch 2 is horizontal and vice versa on neighbouring satellites.

%\item Each station is transmitted in its own band.
\item TV channels transmitted through cable (CATV) are shown as  \psframebox[linestyle=none]{\rput(0.05,.05){\psframebox[fillstyle=solid,framearc=0.1,fillcolor=blue]{\textcolor{white}{\tiny 05}}}}\hspace{.1in}. CATV channels starting with ``T-" are channels fed back to the cable TV station (like news feeds).
\item Air and cable analog TV stations are broadcast with the separate video, colour, and audio frequency carriers grouped together in a channel band as follows:\vspace{.31in}\\
	\psframebox[linestyle=none]{\psset{xunit=1.7in}
		\psline{|<->|}(0,.28)(2.5,.28)\uput{1pt}[90](1.25,.28){6MHz}
		\white
		\psframe[linestyle=solid,framearc=0.25,fillcolor=red,fillstyle=solid](0,0)(2.5,.25)
		\psdots[linecolor=white,dotstyle=triangle*](0.625,0.02)(1.915,0.02)(2.375,0.02)
		\psline[linecolor=white](0.625,0)(0.625,.25)
		\psline[linecolor=white](1.915,0)(1.915,.15)
		\psline[linecolor=white](2.375,0)(2.375,.25)
		\psline[linecolor=white]{<->}(0,.125)(.625,.125)\rput(.29,.125){
			\psframebox[linearc=1pt,linestyle=none,framesep=0pt,fillcolor=white,fillstyle=solid]{\textcolor{Black}{1.25MHz}}}
		\psline[linecolor=white]{<->}(.625,.07)(1.915,.07)\rput(1.27,.07){
			\psframebox[linearc=1pt,linestyle=none,framesep=0pt,fillcolor=white,fillstyle=solid]{\textcolor{Black}{3.58MHz}}}
		\psline[linecolor=white]{<->}(.625,.18)(2.375,.18)\rput(1.50,.18){
			\psframebox[linearc=1pt,linestyle=none,framesep=0pt,fillcolor=white,fillstyle=solid]{\textcolor{Black}{4.5MHz}}}
		\uput{1pt}[270](.625,0){Video}
		\uput{1pt}[270](1.915,0){Colour}
		\uput{1pt}[270](2.375,0){Audio}
		}\vspace{0.1in}

%TV horizontal refresh
\item 15.7 kHz horizontal sweep signal is a common contaminant to VLF listening \hspace{.05in}
  \psframebox[framearc=0,linearc=0]{\rput(0,.05){
	\psframe[cornersize=relative,linecolor=white, linestyle=solid, linewidth=0.8pt,fillstyle=solid,framearc=.25,fillcolor=red,linearc=0.25](-.1,-.1)(.1,.1)
	\psline[linecolor=white,linestyle=solid,linewidth=1pt]{<->}(-.1,0)(.1,0)
  }}\hspace{0.07in}.
\item Digital compression methods are used for HDTV broadcasts in order to pack more channels into the same 6MHz bandwidth as analog TV.

\end{itemize}

}



