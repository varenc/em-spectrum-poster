{\Large Polarization}
\begin{itemize}

\item As a photon (light particle) travels through space, its axis of electrical and magnetic fluctuations does not rotate. Therefore, each photon has a fixed linear polarity of somewhere between 0\degree\ to 360\degree. Light can also be circularly and elliptically polarized.

\item Some crystals can cause the photon to rotate its polarization.

\item Receivers that expect polarized photons will not accept photons that are in other polarities. (ex.\ satellite dish receivers have horizontal and vertical polarity positions).

\item A polarized filter (like Polaroid\texttrademark\  sunglasses) can be used to demonstrate polarized light. One filter will only let photons that have one polarity through. Two overlapping filters at right angles will almost completely block the light that exits; however, a third filter inserted between the first two at a 45\degree\ angle will rotate the polarized light and allow some light to come out the end of all three filters.

\item Light that reflects off an electrical insulator becomes polarized. Conductive reflectors do not polarize light.

% from Doug Welch
\item Perhaps the most reliably polarized light is a rainbow.

\item Moonlight is also slightly polarized. You can test this by viewing the moonlight through a Polaroid\texttrademark sunglass lens, then rotate that lens, the moonlight will dim and brighten slightly.

\end{itemize}


